\documentclass[a4paper,12pt]{article}
\usepackage[top=2cm, bottom=2cm, left=2.5cm, right=2.5cm]{geometry} %margens
\usepackage[utf8]{inputenc}%assim aparecem os acentos


\usepackage[utf8]{inputenc}    
\usepackage[brazil]{babel}
\usepackage{listings}
\usepackage{color}

\definecolor{dkgreen}{rgb}{0,0.6,0}
\definecolor{gray}{rgb}{0.5,0.5,0.5}
\definecolor{mauve}{rgb}{0.58,0,0.82}

\lstset{
  language=Python,                
  basicstyle=\footnotesize,           
  numbers=left,                   
  numberstyle=\tiny\color{gray},  
  stepnumber=1,                             
  numbersep=5pt,                  
  backgroundcolor=\color{white},    
  showspaces=false,               
  showstringspaces=false,         
  showtabs=false,                 
  frame=single,                   
  rulecolor=\color{black},        
  tabsize=2,                      
  captionpos=b,                   
  breaklines=true,                
  breakatwhitespace=false,        
  title=\lstname,                               
  keywordstyle=\color{blue},          
  commentstyle=\color{dkgreen},       
  stringstyle=\color{mauve},     
}

\begin{document}

 


\begin{center}
    \textbf{Métodos numéricos para solução de EDOs}
\end{center}

\begin{center}
    \textbf{Método de Euler:} $y_{n+1} = y_n + f(t_n,y_n)h$
\end{center}


\textbf{Exemplo 1:} $y' = 1 - t + 4y$

\begin{lstlisting}
    import numpy as np
    import matplotlib.pylab as plt
    
    
    def f(t, y):
        return 1 - t + 4*y
    
    
    def EDO_euler(f, y0, t0, NUMBER_OF_STEPS=100, h=0.01):
    
        y = np.zeros(NUMBER_OF_STEPS, dtype=np.float32)
        t = np.zeros(NUMBER_OF_STEPS, dtype=np.float32)
    
        y[0] = 1
        t[0] = 0
    
        for n in range(0, NUMBER_OF_STEPS - 1):
            K1 = f(t[n], y[n])
            y[n+1] = y[n] + K1*h  
            t[n+1] = t[n]+h  
    
        return (t, y)
    
    
    t, y = EDO_euler(f, 1, 0) 
    plt.plot(t, y)
    plt.show()

\end{lstlisting}
\end{document}
